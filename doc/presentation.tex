\documentclass{beamer}
\usepackage{hyperref} % Include this package to use URLs
\usetheme{metropolis}
\setbeamertemplate{section in toc}[sections numbered]
\setbeamertemplate{subsection in toc}[subsections numbered]
\setbeamertemplate{page number in head/foot}{}


\begin{document}

\title{Implementation of a Distributed Bibliographic Search Engine}
\subtitle{\url{http://46.101.232.219:8080/bibengine/}} % Add your URL here
\author{Galanopoulou Rafaila ,
\{Kalopisis, 
 Notaris\}  Ioannis
 }
\date{\today}

\begin{frame}
\titlepage
\end{frame}

\begin{frame}{Table of Contents}
\tableofcontents[currentsection, hideothersubsections]
\end{frame}

\section{Idea}

\begin{frame}{Project Overview}
    We developed a bibliographic search engine:
    \begin{itemize}
    \item Utilizes Crossref dataset (157 GB compressed)
    % \item Covers 134 million publications with full citations
    \item Distributed system with 8 nodes
    \item MongoDB database for optimized query performance
    \item Ranking algorithm based on BM25F
    \item Indexing mechanism for quick search operations
    \end{itemize}
    Integrated machine learning for:
    \begin{itemize}
    \item Alternative query suggestions
    \item Improved user experience and relevance
    \item Continuous adaptation to user needs
    \end{itemize}
    \end{frame}
    
\section{Implementation}

\subsection{Ranking Implementation}

\begin{frame}{Ranking Implementation}
    The ranking process using the BM25F algorithm:
    \begin{enumerate}
    \item BM25F class defines the algorithm, taking the inverted index and document count as input.
    \item bm25f method calculates the BM25F score for documents and a query.
    \item idf calculation method calculates IDF for each query term.
    \item tf field calculation method calculates TF for each term and field.
    \item algorithm parameters method returns field-specific parameters.
    \item SearchEngine class performs the search operation using BM25F.
    \item BooleanInformationRetrieval class performs boolean search.
    \item sort documents method sorts documents based on their scores.
    \end{enumerate}
\end{frame}

\subsubsection{Boolean Information Retrieval Algorithm}

    \begin{frame}{Boolean Information Retrieval Algorithm}
    The Boolean Information Retrieval algorithm:
    \begin{itemize}
    \item Initially narrows down the results based on query words.
    \item Assigns an initial score to documents based on word matches.
    \item Documents with the highest score proceed to the next stage of the BM25F algorithm.
    \end{itemize}
    \end{frame}
    


\subsubsection{Preprocessor Package}
\begin{frame}{Preprocessor Package}
    The preprocessor package provides functionality for data cleaning and preparation:
    
    \begin{itemize}
    \item DataCleaner class performs data cleaning operations, including removing HTML tags, converting to lowercase, tokenizing, and removing stopwords and punctuation.
    \item Functions like \texttt{clean data(data)}, \texttt{html strip(data)}, \texttt{lower string(data)}, \texttt{tokenizer(data)}, and \texttt{punctuation deletion(data)} are available for specific cleaning tasks.
    \item The DataCleaner class uses the NLTK library and downloads necessary resources during initialization.
    \item It prepares data for further analysis or indexing by removing HTML tags, converting to lowercase, tokenizing, and removing stopwords and punctuation.
    \end{itemize}

\end{frame}
    

\subsection{Indexer Package}

\begin{frame}{Indexer Package}
    The indexer package provides functionality to:
    \begin{itemize}
    \item Metadata class: Handles metadata operations.
    \item InfoClass class: Stores document information.
    \item IndexCreator class: Creates and loads the index.
    \item Metadata methods: Init, add doc length, increase average length, calculate average length, add referenced by, normalize referenced by, set total docs, load, save.
    \item InfoClass methods: Convert to dictionary representation.
    \item IndexCreator methods: Init, create/load index, node adder.
    \item IndexCreator utilizes DataCleaner for data cleaning.
    \item Interacts with MongoDB using db.connection module.
    \end{itemize}
    The indexer package ensures efficient retrieval and analysis of indexed data.

\end{frame}

\subsection{APIRequester Package}

\begin{frame}{APIRequester Package}
    The APIRequester package provides functionality to:
    \begin{itemize}
    \item Send HTTP requests to the API Server.
    \item APIRequester class: Handles authentication and POST requests.
    \item Key methods: init, post update config endpoint.
    \item Requires base URL, username, and password.
    \item Utilizes http.client, base64, json, and urllib.parse modules.
    \item Establishes connection and sends HTTP requests.
    \item Parses JSON responses and handles authentication.
    \end{itemize}
    APIRequester simplifies authentication and sending requests.
\end{frame}
    

\subsection{API Server}
\begin{frame}{API Server Package}
    The API server provides functionality to:
    \begin{itemize}
    \item Handle different types of requests.
    \item FastAPI framework implementation.
    \item Endpoints for search, alternate queries, fetch data, and update config.
    \item Supports basic authentication.
    \item Uses ConfigManager and RequestWrapper.
    \item Async calls with ThreadPoolExecutor.
    \item Uvicorn package for running the server.
    \item RESTful API with authentication.
    \item Search, fetch data, and configuration update.
    \end{itemize}
    The API server enables searching, fetching data, and configuration updates.
    \end{frame}
    

\subsection{Distributed Package}

\begin{frame}{Distributed Package Package}
    The distributed package offers functionality to:
    \begin{itemize}
    \item Handle distributed JSON documents.
    \item Shard Calculation: Determine shard ID based on document ID and number of shards.
    \item FastJsonLoader Class: Load and access distributed JSON documents.
    \item MongoDB used for metadata and document IDs.
    \item Methods for loading, saving, retrieving, inserting, and deleting documents.
    \item Efficient management of distributed JSON documents.
    \end{itemize}
    This package simplifies working with distributed JSON documents.
    \end{frame}
    
\subsection{Implementation of a Distributed Bibliographic Search Engine}

\begin{frame}{Distributed Bibliographic Search Engine}
    The implementation of a distributed bibliographic search engine includes:
    \begin{itemize}
    \item Designed for large volumes of bibliographic data.
    \item Multiple nodes distribute workload and improve performance.
    \item DistributedNode class manages operations of a single node.
    \item Handles requests, document loading, indexing, and search.
    \item Heartbeat and initialization process for node communication.
    \item Data retrieval and search based on IDs or query string.
    \item Configuration management for distributed collaboration.
    \item Efficient processing and accurate search results.
    \end{itemize}
    The distributed search engine efficiently handles bibliographic data.
    \end{frame}
    

\section{Suggest Alternative Queries}

\begin{frame}{Suggest Alternative Queries Summary}
    The implementation of alternative query suggestion includes:
    \begin{itemize}
    \item KNN algorithm for predicting synonyms.
    \item \textbf{Word2Vec for encoding word meanings.}
    \item Sent2Vec for generating embeddings of sentences.
    \item BERT for advanced word embeddings.
    \item T5 model for text-to-text transfer tasks.
    \item Transformers for contextual understanding of text.
    \item Limitations:
    \begin{itemize}
        \item Resource-intensive and opaque nature of models.
        \item Biases and limitations in contextual understanding.
        \item Challenges in training on new tasks.
        \item Potential risks of exposing sensitive information.
    \end{itemize}
    \end{itemize}
    These techniques enhance alternative query generation.
    \end{frame}
    

\section{Deployment}

\begin{frame}{Deployment}
    For our bibliographic search engine deployment:
    \begin{itemize}
    \item Front-end developed with Spring Boot.
    \item Back-end implemented in Python.
    \item Automated testing for code quality.
    \item Continuous integration and deployment pipelines.
    \item Eight-node distributed database approach.
    \item Enhanced performance and fault tolerance.
    \item High volume query handling capability.
    \item Detailed hardware specifications in Appendix A.
    \end{itemize}
    This approach ensures efficient data processing.
\end{frame}
    
\section{Observations}

\begin{frame}{Observations 1/2}
    \begin{itemize}
        \item Inverted Index:
        \begin{itemize}
        \item Efficient data structure for full-text search indexing.
        \item Lists documents for each term instead of listing terms for each document.
        \item Enables fast keyword-based searches by looking up keywords in the index.
        \item Avoids scanning through all documents in a collection for keyword search.
        \end{itemize}
        \item Trie Data Structure:
        \begin{itemize}
        \item Used to build the inverted index.
        \item Dramatically reduces index size.
        \item Enables the index to fit entirely in memory.
        \item Improves search speed.
        \end{itemize}
        \end{itemize}
\end{frame}

\begin{frame}{Observations 1/2}
    \begin{itemize}
        \item Inverted Index:
        \begin{itemize}
        \item Efficient data structure for full-text search indexing.
        \item Lists documents for each term instead of listing terms for each document.
        \item Enables fast keyword-based searches by looking up keywords in the index.
        \item Avoids scanning through all documents in a collection for keyword search.
        \end{itemize}
        \item Trie Data Structure:
        \begin{itemize}
        \item Used to build the inverted index.
        \item Dramatically reduces index size.
        \item Enables the index to fit entirely in memory.
        \item Improves search speed.
        \end{itemize}
        \end{itemize}
\end{frame}

\begin{frame}{Observations 2/2}
\begin{itemize}
    \item BM25F Parallelization:
    \begin{itemize}
    \item Attempted parallelization of BM25F algorithm.
    \item Explored process-level and thread-level parallelization techniques.
    \item Did not observe performance improvements due to overhead and Global Interpreter Lock (GIL) limitations in Python.
    \item CPU-intensive nature of the algorithm hindered multi-threading benefits.
    \item Multi-process techniques could be explored with different CPU architecture and larger data volumes for potential speed improvements.
    \end{itemize}
    \end{itemize}
\end{frame}
\end{document}
